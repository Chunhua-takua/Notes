\newpage
\section{二心集}

\setcounter{enumi}{0}

\subsection{“好政府主义”}
\item {
    \CJKfamily{chisugafont}
    有如被压榨得痛了,就要叫喊,原不必在想出更好的主义之前,就定要咬住牙关。

    他在药方上所开的却不是药名,而是“好药料”三个大字。
}

\subsection{“丧家的”“资本家的乏走狗”}
\item {
    \CJKfamily{chisugafont}
    凡走狗,虽或为一个资本家所豢养,其实是属于所有的资本家的,所以它遇见所有的阔人都驯良,遇见所有的穷人都狂吠。不知道谁是它的主子,正是它遇见所有阔人都驯良的原因,也就是属于所有的资本家的证据。即使无人豢养,饿的精瘦,变成野狗了,但还是遇见所有的阔人都驯良,遇见所有的穷人都狂吠的,不过这时它就愈不明白谁是主子了。
}

\subsection{“硬译”与“文学的阶级性”}
\item {
    \CJKfamily{chisugafont}
    至于无产者应该“辛辛苦苦”爬有产阶级去的“正当”的方法,则是中国有钱的老太爷高兴时候,教导穷工人的古训,在实际上,现今正在“辛辛苦苦诚诚实实”想爬上一级去的“无产者”也还多。
}

\subsection{“友邦惊诧”论}
\item {
    \CJKfamily{chisugafont}
    “友邦人士,莫名惊诧,长此以往,国将不国”了!
}

\subsection{“智识劳动者”万岁}
\item {
    \CJKfamily{chisugafont}
    “劳动者”这句话成了“罪人”的代名词,已经足足四年了。压迫罢,谁也不响;杀戮罢,谁也不响;文学上一提起这句话,就有许多“文人学士”和“正人君子”来笑骂,接送又有许多他们的徒子徒孙来笑骂。劳动者呀劳动者,真要永世不得翻身了。
}

\subsection{沉滓的泛起}
\item {
    \CJKfamily{chisugafont}
    要趁“国难声中”或“和平声中”将利益更多的榨到自己的手里的。
}

\subsection{对于左翼作家联盟的意见}
\item {
    \CJKfamily{chisugafont}
    倘若不和实际的社会斗争接触,单关在玻璃窗内做文章,研究问题,那是无论怎样的激烈,“左”,都是容易办到的;然而一碰到实际,便即刻要撞碎了。

    现在为劳动大人众革命,将来革命成功,劳动阶级一定从丰报酬,特别优待,请他坐特等车,吃特等饭,或者劳动者捧着牛油面包来献他,说:“我们的诗人,请用吧!”这也是不正确的;因为实际上决不会有这种事,恐怕那时比现在还要苦,不但没有牛油面包,连黑面包都没有也说不定,俄国革命后一二年的情形便是例子。

    “反动派且已经有联合战线了,而我们还没有团结起来!”只因为他们的目的相同,所以行动就一致,在我们 看来就好像联合战线。
}

\subsection{非革命的急进革命论者}
\item {
    \CJKfamily{chisugafont}
    他现为批评家而说话的时候,就随便捞到一种东西以驳诘相反的东西。要驳互助说时用争存说,驳争存说时用互助说;反对和平论时用阶级争斗说,反对斗争时就主张人类之爱。论敌是唯心论者呢,他的立场是唯物论,待到和唯物论者相辩难,他却又化为唯心论者了。
}

\subsection{关于翻译的通信}
\item {
    \CJKfamily{chisugafont}
    不可与言而与之言,失言。

    这正如俄国革命以后,欧美的富家奴去看了一看,回来就摇头皱脸,做出文章,慨叹着工农还在怎样吃苦,怎样忍饥,说得满纸凄凄惨惨。仿佛惟有他却是极希望一个筋斗,工农就都住王宫,吃大菜,躺安乐椅子享福的人。谁料还是苦,所以俄国不行了,革命不好了,阿呀阿呀了,可恶之极了。
}

\subsection{黑暗中国的文艺界的现状}
\item {
    \CJKfamily{chisugafont}
    他以为文艺原不是无产阶级的东西,无产者倘要创作或鉴赏文艺,先应该辛苦地积钱,爬上资产阶级去,而不应该大家浑身褴褛,到这花园中来吵嚷。
}

\subsection{上海文艺之一瞥}
\item {
    \CJKfamily{chisugafont}
    去嫖的时候,可以叫十个二十个的年青姑娘聚集在一处,样子很有些像《红楼梦》,于是他就觉得自己好像贾宝玉;自己是才子,那么婊子当然是佳人,于是才子佳人的书就产生了。

    佳人并非因为“爱才若渴”而做婊子的,佳人只为的是钱。

    其实革命是并非教人死而是教人活的。

    要人帮忙时候用克鲁巴金的互助论,要和人争闹的时候就用达尔文的生存竞争说。
}

\subsection{唐朝的钉梢}
\item {
    \CJKfamily{chisugafont}
    即使骂,也就大有希望,因为一骂便可有言语来往,所以也就是“扳谈”的开头。
}

\subsection{现代电影与有产阶级}
\item {
    \CJKfamily{chisugafont}
    在实际上,电影是以大多数的小市民和无产阶级为看客的。而他们,小市民和无产阶级,早已渐渐地觉察出有产阶级的诡计来了。就是,已经注意于“支配阶级制作了宣布那服从于己的观念形态的影片,而以此来做掠取无产者的衣袋的手段”这事实的真相了。

    拙劣的煽动,却招致反对的结果。

    露骨的宣传是停止了。最所希望的,是使电影的看客看不见“阶级”这观念。至少,是坐在银幕之前的数小时中,使他们忘却了一切社会底对立。
    这样子,就产生了小市民的影片。

    过屠门与大嚼,虽不得肉,亦且快意。
}

\subsection{序言}
\item {
    \CJKfamily{chisugafont}
    在坏了下去的旧社会里,倘有人怀一点不同的意见,有一点携贰的心思,是一定要吃其苦的。而攻击陷害得最凶的,则是这人的同阶级的人物。
}

\subsection{宣传与做戏}
\item {
    \CJKfamily{chisugafont}
    全国的人们十之九不识字,然而总得请几位博士,使他对西洋人去讲中国的精神文明。
}

\subsection{中国无产阶级革命文学和前驱的血}
\item {
    \CJKfamily{chisugafont}
    我们的劳苦大众历来只被最剧烈的压迫和榨取,连识字教育的布施也得不到,惟有默默地身受着宰割和灭亡。繁难的象形字,又使他们不能有自修的机会。
}