\newpage
\section{而已集}

\setcounter{enumi}{0}

% \renewcommand{\baselinestretch}{3.0}

\subsection{“公理”之所在}
\item {
    \CJKfamily{chisugafont}
    我的话已经说完,去年说的,今年还适用,恐怕明年也还适用。但我诚恳地希望他不至于适用到十年二十年之后。
}

\subsection{答有恒先生}
\item {
    \CJKfamily{chisugafont}
    我也在救助我自己,还是老法子:一是麻痹,二是忘却。
}

\subsection{读所谓“大内档案”}
\item {
    \CJKfamily{chisugafont}
    军阀是不看杂志的,就靠叭儿狗嗅,候补叭儿狗吠。
}

\subsection{革命时代的文学}
\item {
    \CJKfamily{chisugafont}
    但为什么人类成了人,猴子终于是猴子呢?这就因为猴子不肯变化——它爱用四只脚走路。也许曾有一个猴子站起来,试用两脚走路的罢,但许多猴子就说:“我们底祖先一向是爬的,不许你站!”咬死了。

    有些民族因为叫苦无用,连苦也不叫了。

    所有的文学,歌呀,诗呀,大抵是给上等人看的:他们吃饭了,睡在躺椅上,捧着看。

    或者讲上等人怎样有趣和快乐,下等人怎样可笑。
    
    下等人没奈何,也只好替他们一同欢喜欢喜。
}

\subsection{黄花节的杂感}
\item {
    \CJKfamily{chisugafont}
    久受压制的人们,被压制时只能忍苦,幸而解放了便只知道作乐,悲壮剧是不能久留在记忆里的。

    然而革命成功的时候,革命家死掉了,却能每年给生存的大家以热闹,甚而至于欢欣鼓舞。惟独革命家,无论他生或死,都能给大家以幸福。
}

\subsection{略谈香港}
\item {
    \CJKfamily{chisugafont}
    其实是种族革命,要将土地从异族的手里取得,归还旧主人。
}

\subsection{谈“激烈”}
\item {
    \CJKfamily{chisugafont}
    因为奴才都叹气,虽无大害,主人看了究竟不舒服。必须要如罗素所称赞的杭州的轿夫一样,常是笑嘻嘻。
}

\subsection{小杂感}
\item {
    \CJKfamily{chisugafont}
    约翰穆勒说:专制使人们变成冷嘲。

    而他竟不知道共和使人们变成沉默。
    
    \vspace{1em}
    曾经阔气的要复古,正在阔气的要保持现状,未曾阔气的要革新。

    大抵如是。大抵!

    \vspace{1em}
    人类的悲欢并不相通,我只觉得他们吵闹。

    \vspace{1em}
    叭儿狗往往比它的主人更严厉。

    \vspace{1em}
    凡为当局所“诛”者皆有“罪”。

    \vspace{1em}
    法三章者,话一句耳。
}

\subsection{忽然想到}
\item {
    \CJKfamily{chisugafont}
    约翰穆勒说:专制使人们变成冷嘲。

    而他竟不知道共和使人们变成沉默。
    
    \vspace{1em}
    曾经阔气的要复古,正在阔气的要保持现状,未曾阔气的要革新。

    大抵如是。大抵!

    \vspace{1em}
    人类的悲欢并不相通,我只觉得他们吵闹。

    \vspace{1em}
    叭儿狗往往比它的主人更严厉。

    \vspace{1em}
    凡为当局所“诛”者皆有“罪”。

    \vspace{1em}
    法三章者,话一句耳。

    \vspace{1em}
    我觉得革命以前,我是奴隶;革命以后不多久,就受了奴隶的骗,变成他们的奴隶了。
    
    我觉得许多烈士的血都被人们踏灭了,然而又不是故意的。

    \vspace{1em}
    试将记五代,南宋,明末的事情的,和现今的状况一比较,就当惊心动魄于何其相似之甚,仿佛时间的流驶,独与我们中国无关。

    “地大物博,人口众多”,用了这许多好材料,难道竟不过老是演一出轮回把戏而已么?

    \vspace{1em}
    有些外人,很希望中国永是一个大古董以供他们的赏鉴,这虽然可恶,却还不奇,因为他们究竟是外人。而中国竟也有自己还不够,并且要率领了少年,赤子,共成一个大古董以供他们的赏鉴者,则真不知是生着怎样的心肝。
    
    \vspace{1em}
    可惜中国人但对于羊显凶兽相,而对于凶兽则显羊相,所以即使显着凶兽相,也还是卑怯的国民。这样下去,一定要完结的。

    对手如凶兽时就如凶兽,对手如羊时就如羊!
    
    \vspace{1em}
    做事的总不如做文的有名。

    新的生命就会在这苦痛的沉默里萌芽。
}