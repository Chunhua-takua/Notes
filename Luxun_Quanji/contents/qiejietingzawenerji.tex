\newpage
\section{且介亭杂文二集}

\setcounter{enumi}{0}

\subsection{序言}
\item {
    \CJKfamily{chisugafont}
    我有时决不想在言论界求得胜利,因为我的言论有时是枭鸣,报告着大不吉利事,我的言中,是大家会有不幸的。
}

\subsection{隐士}
\item {
    \CJKfamily{chisugafont}
    陶渊明先生是我们中国赫赫有名的大隐,一名``田园诗人'',自然,他并不办期刊,也赶不上吃``庚款'',然而他有奴子。汉晋时候的奴子,是不但侍候主人,并且给主人,并且给主人种地,营商的,正是生财器具。

    至于那些文士诗翁,自称什么钓徒樵子的,倒大抵是悠游自得的封翁或公子,何尝捏过钓竿或斧头柄。
}

\subsection{``招贴即扯''}
\item {
    \CJKfamily{chisugafont}
    揭穿假面,就是指出了实际来,这不能混谓之骂。
}

\subsection{内山完造《活中国的姿态》序}
\item {
    \CJKfamily{chisugafont}
    他一看见本国里乞丐之多,非常诧异,慨叹道:他们为什么不去研究学问,却自甘堕落的呢?所以下等人实在是无可救药的。
}

\subsection{论讽刺}
\item {
    \CJKfamily{chisugafont}
    有好些直写事实的作者,就这样的被蒙上了``讽刺家''——很难说是好是坏——的头衔。
}

\subsection{从``别字''说开去}
\item {
    \CJKfamily{chisugafont}
    我以为方块字本身就是一个死症,吃点人参,或者想一点什么方法,固然也许可以拖延一下,然而到底是无可挽救的,所以一向就不大注意这回事。
}

\subsection{田军作《八月的乡村》序}
\item {
    \CJKfamily{chisugafont}
    ``若要官,杀人放火受招安;若要富,跟着行在卖酒醋。''

    人民在欺骗和压制之下,失了力量,哑了声音,至多也不过有几句民谣。``天下有道,则庶人不议。''
}

\subsection{徐懋庸作《打杂集》序}
\item {
    \CJKfamily{chisugafont}
    这捧了起来,却不过为了接着摔得粉碎。大约还有人记得``美人鱼''罢,科捧得令观者发生肉麻之感,连看见姓名也会觉得有些滑稽。
}


%%%%%
\subsection{在现代中国的孔夫子}
\item {
    \CJKfamily{chisugafont}
    我出世的时候是清朝的末年,孔夫子已经有了``大成至圣文宣王''这一个阔得可怕的头衔,不消说,正是圣道支配了全国的时代。政府对于读书的人们,使读一定的书,即四书和五经;使遵守一定的注释;使写一定的文章,即所谓``八股文'';并且使发一定的议论。

    总而言之,孔夫子在中国,是权势者们捧起来的,是那些权势者或想做权势者们的圣人,和一般的民众并无什么关系。

    他们都是连字也不大认识的人物,然而偏要大谈什么《十三经》之类,所以使人们觉得滑稽;言行也太不一致了,就更加令人讨厌。

    所以把孔子装饰得十分尊严时,就一定有找他缺点的论文和作品出现。

    不错,孔夫子曾经计划过出色的治国的方法,但那都是为了治民众者,即权势者设想的方法,为民众本身的,却一点也没有。这就是``礼不下庶人''。
}

\subsection{论``人言可畏''}
\item {
    \CJKfamily{chisugafont}
    她们的死,不过像在无边的人海里添了几粒盐,虽然使扯淡的嘴巴们觉得有些味道,但不久也还是淡,淡,淡。

    对强者它是弱者,但对更弱者它却还是强者,所以有时虽然吞声忍气,有时仍可以耀武扬威。

    小市民总爱听人们的丑闻,尤其是有些熟识的人的丑闻。

    化几个铜元就发见了自己的优胜,那当然是很上算的。
}

\subsection{再论``文人相轻''}
\item {
    \CJKfamily{chisugafont}
    这自然也许未必全无好处,但做文人做到这地步,不是很有些近乎婊子了么?
}

\subsection{文坛三户}
\item {
    \CJKfamily{chisugafont}
    要之,凡有弄弄笔墨的人们,他先前总有一点凭借:不是祖遗的正在少下去的钱,就是父积的还在多起来的钱。要不然,他就无缘读书识字。
}

\subsection{名人与名言}
\item {
    \CJKfamily{chisugafont}
    悖在倚专家之名,来论他所专门以外的事。社会上崇敬名人,于是以为名人的话就是名言,却忘记了他之所以得名是那一种学问或事业。
}

\subsection{``靠天吃饭''}
\item {
    \CJKfamily{chisugafont}
    大约是西洋人说的罢,世界上穷人有份的,只有日光空气和水。这在现在的上海就不适用,卖心卖力的被一天关到夜,他就晒不着日光,吸不到好空气;装不起自来水的,也喝不到干净水。报上往往说:``近来天时不正,疾病盛行'',这岂只是``天时不正''之故,``天何言哉'',它默默地被冤枉了。
}