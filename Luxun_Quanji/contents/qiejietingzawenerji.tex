\newpage
\section{且介亭杂文二集}

\setcounter{enumi}{0}

\subsection{序言}
\item {
    \CJKfamily{chisugafont}
    我有时决不想在言论界求得胜利,因为我的言论有时是枭鸣,报告着大不吉利事,我的言中,是大家会有不幸的。
}

\subsection{隐士}
\item {
    \CJKfamily{chisugafont}
    陶渊明先生是我们中国赫赫有名的大隐,一名``田园诗人'',自然,他并不办期刊,也赶不上吃``庚款'',然而他有奴子。汉晋时候的奴子,是不但侍候主人,并且给主人,并且给主人种地,营商的,正是生财器具。

    至于那些文士诗翁,自称什么钓徒樵子的,倒大抵是悠游自得的封翁或公子,何尝捏过钓竿或斧头柄。
}

\subsection{``招贴即扯''}
\item {
    \CJKfamily{chisugafont}
    揭穿假面,就是指出了实际来,这不能混谓之骂。
}

\subsection{内山完造《活中国的姿态》序}
\item {
    \CJKfamily{chisugafont}
    他一看见本国里乞丐之多,非常诧异,慨叹道:他们为什么不去研究学问,却自甘堕落的呢?所以下等人实在是无可救药的。
}

\subsection{论讽刺}
\item {
    \CJKfamily{chisugafont}
    有好些直写事实的作者,就这样的被蒙上了``讽刺家''——很难说是好是坏——的头衔。
}

\subsection{从``别字''说开去}
\item {
    \CJKfamily{chisugafont}
    我以为方块字本身就是一个死症,吃点人参,或者想一点什么方法,固然也许可以拖延一下,然而到底是无可挽救的,所以一向就不大注意这回事。
}

\subsection{田军作《八月的乡村》序}
\item {
    \CJKfamily{chisugafont}
    ``若要官,杀人放火受招安;若要富,跟着行在卖酒醋。''

    人民在欺骗和压制之下,失了力量,哑了声音,至多也不过有几句民谣。``天下有道,则庶人不议。''
}

\subsection{徐懋庸作《打杂集》序}
\item {
    \CJKfamily{chisugafont}
    这捧了起来,却不过为了接着摔得粉碎。大约还有人记得``美人鱼''罢,科捧得令观者发生肉麻之感,连看见姓名也会觉得有些滑稽。
}


%%%%%
\subsection{关于新文字}
\item {
    \CJKfamily{chisugafont}
    方块汉字真是愚民政策的利器,不但劳苦大众没有学习和学会的可能,就是有钱有势的特权阶级,费时一二十年,终于学不会的也多得很。……不过他们可以装作懂得的样子,来胡说八道,欺骗不明真相的人。
}

\subsection{病后杂谈}
\item {
    \CJKfamily{chisugafont}
    其实,``君子远疱厨也''就是自欺欺人的办法:君子非吃牛肉不可,然而他慈悲,不忍见牛的临死的觳觫,于是走开,等到烧成牛排,然后慢慢的来咀嚼。
}

\subsection{病后杂谈之余}
\item {
    \CJKfamily{chisugafont}
    中国人是一向被同族和异族屠戮,奴隶,敲掠,刑辱,压迫下来的,非人类所能忍受的楚毒,也都身受过,每一考查,真教人觉得不像活在人间。
}

\subsection{附记}
\item {
    \CJKfamily{chisugafont}
    老百姓整千整万地做了炮灰,各国资本家却可以聚首一堂举着香槟相视而笑。
}