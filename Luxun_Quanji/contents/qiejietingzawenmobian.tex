\newpage
\section{且介亭杂文末编}

\setcounter{enumi}{0}

\subsection{我要骗人}
\item {
    \CJKfamily{chisugafont}
    中国的人民是多疑的。无论那一国人,都指这为可笑的缺点。然而怀疑并不是缺点。总是疑,而并不下断语,这才是缺点。

    中国的人民,是常用自己的血,去洗权力者的手,使他又变成洁净的人物的。
}

\subsection{续记}
\item {
    \CJKfamily{chisugafont}
    中国原是``把人不当人''的地方,即使无端诬人为投降或转变,国贼或汉奸,社会上也并不以为奇怪。
}

\subsection{我的第一个师父}
\item {
    \CJKfamily{chisugafont}
    中国的邪鬼,是怕斩钉截铁,不能含胡的东西的。

    ``和尚没有老婆,小菩萨那里来!?''
}

\subsection{半夏小集}
\item {
    \CJKfamily{chisugafont}
    用笔和舌,将沦为异族的奴隶之苦告诉大家,自然是不错的,但要十分小心,不可使大家得着这样的结论:``那么,到底还不如我们似的做自己人的奴隶好。''

    ``明言着轻蔑什么人,并不是十足的轻蔑。惟沉默是最高的轻蔑。''

    最高的轻蔑是无言,而且连眼珠也不转过去。
}

\subsection{死}
\item {
    \CJKfamily{chisugafont}
    大约我们的生死久已被人们随意处置,认为无足重轻,所以自己也看得随随便便,不像欧洲人那样的认真了。有些外国人说,中国最怕死。这其实是不确的,——但自然,每不免模模胡胡的死掉则有之。
}

\subsection{``立此存照''(七)}
\item {
    \CJKfamily{chisugafont}
    不过我们站在中国人的立场上,却还``希望''我们对于自己,也有这``大国民的风度'',不要把自国的人民的生命价值,估计得只值外侨的一半,以至于``罪加一等''。主杀奴无罪,奴杀主重办的刑律,自从民国以来(呜呼,二十五年了!)不是早经废止了么?
}
