\newpage
\section{集外集}

\setcounter{enumi}{0}

\subsection{爱之神}
\item {
    \CJKfamily{chisugafont}
    你要是爱谁,便没命的去爱他;

    你要是谁也不爱,也可以没命的去自己死掉。
}

\subsection{烽话五则}
\item {
    \CJKfamily{chisugafont}
    伶俐人叹``人心不古''时,大抵是他的巧计失败了。
}

\subsection{杂语}
\item {
    \CJKfamily{chisugafont}
    无论谁胜,地狱至今也还是照样的地狱。

    您的少说话就是高深,您的少作文就是名贵,永远不会失败了。
}

\subsection{俄文译本《阿Q正传》序及著者自叙传略}
\item {
    \CJKfamily{chisugafont}
    在我自己,总仿佛觉得我们人人之间各有一道高墙,将各个分离,使大家的心无从相印。这就是我们古代的聪明人,即所谓圣贤,将人们分为十等,说是高下各不相同。

    造化生人,已经非常巧妙,使一个人不会感到别人的肉体上的痛苦了,我们的圣人和圣人之徒却又补了造化之缺,并且使人们不再会感到别人的精神上的痛苦。
}

\subsection{文艺与政治的歧途}
\item {
    \CJKfamily{chisugafont}
    政治家最不喜欢人们反抗他的意见,最不喜欢人家要想,要开口。而从前的社会也的确没有人想过什么,又没有人开过口。

    从生活窘迫过来的人,一到了有钱,容易变成两种情形:一种是理想世界,替处同一境遇,便成为人道主义;一种是什么都是自己挣起来,从前的遭遇,使他觉得什么都是冷酷,便流为个人主义。

    战争的结果,也可以变成两种态度:一种是英雄,他见别人死的死伤的伤,只有他健存,自己就觉得怎样了不得,这么那么夸耀战场上的威雄。一种是变成反对战争的,希望世界上不要再打仗了。

    (托尔斯泰)主张无抵抗主义,叫兵士不替皇帝打仗,警察不替皇帝执法,审判官不替皇帝裁判,大家都不去捧皇帝;皇帝是全要人捧的,没有人捧,还成什么皇帝,更和政治相冲突。

    但做文学的人总得闲定一点,正在革命中,那有功夫做文学。我们且想想:在生活困乏中,一面拉车,一面``之乎者也'',到底不大便当。古人虽有种田做诗的,那一定不是自己在种田;雇了几个人替他种田,他才能吟他的诗;真要种田,就没有功夫做诗。

    革命成功以后,闲空了一点;有人恭维革命,有人颂扬革命,这已不是革命文学。他们恭维革命颂扬革命,就是颂扬有权力者,和革命有什么关系?
}




%%%%%
\subsection{沉滓的泛起}
\item {
    \CJKfamily{chisugafont}
    要趁“国难声中”或“和平声中”将利益更多的榨到自己的手里的。
}

\subsection{对于左翼作家联盟的意见}
\item {
    \CJKfamily{chisugafont}
    倘若不和实际的社会斗争接触,单关在玻璃窗内做文章,研究问题,那是无论怎样的激烈,“左”,都是容易办到的;然而一碰到实际,便即刻要撞碎了。

    现在为劳动大人众革命,将来革命成功,劳动阶级一定从丰报酬,特别优待,请他坐特等车,吃特等饭,或者劳动者捧着牛油面包来献他,说:“我们的诗人,请用吧!”这也是不正确的;因为实际上决不会有这种事,恐怕那时比现在还要苦,不但没有牛油面包,连黑面包都没有也说不定,俄国革命后一二年的情形便是例子。

    “反动派且已经有联合战线了,而我们还没有团结起来!”只因为他们的目的相同,所以行动就一致,在我们 看来就好像联合战线。
}

\subsection{非革命的急进革命论者}
\item {
    \CJKfamily{chisugafont}
    他现为批评家而说话的时候,就随便捞到一种东西以驳诘相反的东西。要驳互助说时用争存说,驳争存说时用互助说;反对和平论时用阶级争斗说,反对斗争时就主张人类之爱。论敌是唯心论者呢,他的立场是唯物论,待到和唯物论者相辩难,他却又化为唯心论者了。
}

\subsection{关于翻译的通信}
\item {
    \CJKfamily{chisugafont}
    不可与言而与之言,失言。

    这正如俄国革命以后,欧美的富家奴去看了一看,回来就摇头皱脸,做出文章,慨叹着工农还在怎样吃苦,怎样忍饥,说得满纸凄凄惨惨。仿佛惟有他却是极希望一个筋斗,工农就都住王宫,吃大菜,躺安乐椅子享福的人。谁料还是苦,所以俄国不行了,革命不好了,阿呀阿呀了,可恶之极了。
}

\subsection{黑暗中国的文艺界的现状}
\item {
    \CJKfamily{chisugafont}
    他以为文艺原不是无产阶级的东西,无产者倘要创作或鉴赏文艺,先应该辛苦地积钱,爬上资产阶级去,而不应该大家浑身褴褛,到这花园中来吵嚷。
}

\subsection{上海文艺之一瞥}
\item {
    \CJKfamily{chisugafont}
    去嫖的时候,可以叫十个二十个的年青姑娘聚集在一处,样子很有些像《红楼梦》,于是他就觉得自己好像贾宝玉;自己是才子,那么婊子当然是佳人,于是才子佳人的书就产生了。

    佳人并非因为“爱才若渴”而做婊子的,佳人只为的是钱。

    其实革命是并非教人死而是教人活的。

    要人帮忙时候用克鲁巴金的互助论,要和人争闹的时候就用达尔文的生存竞争说。
}

\subsection{唐朝的钉梢}
\item {
    \CJKfamily{chisugafont}
    即使骂,也就大有希望,因为一骂便可有言语来往,所以也就是“扳谈”的开头。
}

\subsection{现代电影与有产阶级}
\item {
    \CJKfamily{chisugafont}
    在实际上,电影是以大多数的小市民和无产阶级为看客的。而他们,小市民和无产阶级,早已渐渐地觉察出有产阶级的诡计来了。就是,已经注意于“支配阶级制作了宣布那服从于己的观念形态的影片,而以此来做掠取无产者的衣袋的手段”这事实的真相了。

    拙劣的煽动,却招致反对的结果。

    露骨的宣传是停止了。最所希望的,是使电影的看客看不见“阶级”这观念。至少,是坐在银幕之前的数小时中,使他们忘却了一切社会底对立。
    这样子,就产生了小市民的影片。

    过屠门与大嚼,虽不得肉,亦且快意。
}

\subsection{序言}
\item {
    \CJKfamily{chisugafont}
    在坏了下去的旧社会里,倘有人怀一点不同的意见,有一点携贰的心思,是一定要吃其苦的。而攻击陷害得最凶的,则是这人的同阶级的人物。
}

\subsection{宣传与做戏}
\item {
    \CJKfamily{chisugafont}
    全国的人们十之九不识字,然而总得请几位博士,使他对西洋人去讲中国的精神文明。
}

\subsection{中国无产阶级革命文学和前驱的血}
\item {
    \CJKfamily{chisugafont}
    我们的劳苦大众历来只被最剧烈的压迫和榨取,连识字教育的布施也得不到,惟有默默地身受着宰割和灭亡。繁难的象形字,又使他们不能有自修的机会。
}