\newpage
\section{集外集}

\setcounter{enumi}{0}

\subsection{爱之神}
\item {
    \CJKfamily{chisugafont}
    你要是爱谁,便没命的去爱他;

    你要是谁也不爱,也可以没命的去自己死掉。
}

\subsection{烽话五则}
\item {
    \CJKfamily{chisugafont}
    伶俐人叹``人心不古''时,大抵是他的巧计失败了。
}

\subsection{杂语}
\item {
    \CJKfamily{chisugafont}
    无论谁胜,地狱至今也还是照样的地狱。

    您的少说话就是高深,您的少作文就是名贵,永远不会失败了。
}

\subsection{俄文译本《阿Q正传》序及著者自叙传略}
\item {
    \CJKfamily{chisugafont}
    在我自己,总仿佛觉得我们人人之间各有一道高墙,将各个分离,使大家的心无从相印。这就是我们古代的聪明人,即所谓圣贤,将人们分为十等,说是高下各不相同。

    造化生人,已经非常巧妙,使一个人不会感到别人的肉体上的痛苦了,我们的圣人和圣人之徒却又补了造化之缺,并且使人们不再会感到别人的精神上的痛苦。
}

\subsection{文艺与政治的歧途}
\item {
    \CJKfamily{chisugafont}
    政治家最不喜欢人们反抗他的意见,最不喜欢人家要想,要开口。而从前的社会也的确没有人想过什么,又没有人开过口。

    从生活窘迫过来的人,一到了有钱,容易变成两种情形:一种是理想世界,替处同一境遇,便成为人道主义;一种是什么都是自己挣起来,从前的遭遇,使他觉得什么都是冷酷,便流为个人主义。

    战争的结果,也可以变成两种态度:一种是英雄,他见别人死的死伤的伤,只有他健存,自己就觉得怎样了不得,这么那么夸耀战场上的威雄。一种是变成反对战争的,希望世界上不要再打仗了。

    (托尔斯泰)主张无抵抗主义,叫兵士不替皇帝打仗,警察不替皇帝执法,审判官不替皇帝裁判,大家都不去捧皇帝;皇帝是全要人捧的,没有人捧,还成什么皇帝,更和政治相冲突。

    但做文学的人总得闲定一点,正在革命中,那有功夫做文学。我们且想想:在生活困乏中,一面拉车,一面``之乎者也'',到底不大便当。古人虽有种田做诗的,那一定不是自己在种田;雇了几个人替他种田,他才能吟他的诗;真要种田,就没有功夫做诗。

    革命成功以后,闲空了一点;有人恭维革命,有人颂扬革命,这已不是革命文学。他们恭维革命颂扬革命,就是颂扬有权力者,和革命有什么关系?
}

\subsection{自嘲}
\item {
    \CJKfamily{chisugafont}
    运交华盖欲何求,未敢翻身已碰头。

    破帽遮颜过闹市,漏船载酒泛中流。

    横眉冷对千夫指,俯首甘为孺子牛。

    躲进小楼成一统,管他冬夏与春秋。
}

\subsection{二十二年元旦}
\item {
    \CJKfamily{chisugafont}
    到底不如租界好,打牌声里又新春。
}

\subsection{题三义塔}
\item {
    \CJKfamily{chisugafont}
    精禽梦觉仍衔石,斗士诚坚共抗流。

    度尽劫波兄弟在,相逢一笑泯恩仇。
}