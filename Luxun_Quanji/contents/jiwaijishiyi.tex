\newpage
\section{集外集拾遗}

\setcounter{enumi}{0}

\subsection{诗歌之敌}
\item {
    \CJKfamily{chisugafont}
    他对我们喜欢说洋话,使我不知所云,然而看见洋人却常说中国话。这记忆忽然给我一种启示,我就想在《文学周刊》上论打拳;至于诗呢,留待将来遇见拳师的时候再讲。

    ``诗人就像赛跑的马,所以应该给吃一点好东西。但不可使他们太肥;太肥,他们就不中用了。''
}

\subsection{报《奇哉所谓……》}
\item {
    \CJKfamily{chisugafont}
    我还只愿意和外国以宾主关系相通,不忍见再如五胡乱华以至满洲入关那样,先以主奴关系而后有所谓``同化''!

    我以为如果外国人来灭中国,是只教你略能说几句外国话,却不至于劝你多读外国书,因为那书是来灭的人们所读的。但是还要奖励你多读中国书,孔子也还要更加崇奉,像元朝和清朝一样。
}

\subsection{这是这么一个意思}
\item {
    \CJKfamily{chisugafont}
    先是小喝,继而大喝,可是酒量愈增,食量就减下去了,我知道酒精已经害了肠胃。现在有时戒除,有时也还喝,正如还要翻翻中国书一样。但是和青年谈起包含来,我总说:你不要喝酒。听的人虽然知道我曾经纵酒,而都明白我的意思。
}

\subsection{中山先生逝世后一周年}
\item {
    \CJKfamily{chisugafont}
    据说当西医已经束手的时候,有人主张服中国药了;但中山先生不赞成,以为中国的药品固然也有有效的,诊断的知识却缺如。不能诊断,如何用药?毋须服。人当濒危之际,大抵是什么也肯尝试的,而他对于自己的生命,也仍有这样分明的理智和坚定的意志。
}

\subsection{《争自由的波浪》小引}
\item {
    \CJKfamily{chisugafont}
    英雄的血,始终是无味的国土里的人生盐,而且大抵是给闲人们作生活的盐,这倒实在是很可诧异的。
}

\subsection{老调子已经唱完}
\item {
    \CJKfamily{chisugafont}
    中国人有一种矛盾思想,即是:要子孙生存,而自己也想活得长久,永远不死;及至知道没法可想,非死不可了,却希望自己的尸身永远不腐烂。

    所以,我想,凡是老的,旧的,实在倒不如高高兴兴的死去的好。

    有人说:``可见中国的老调子实在好,正不妨唱下去。试看元朝的蒙古人,清朝的满洲人,不是都被我们同化了么?照此看来,则将来无论何国,中国都会这样地将他们同化的。''原来我们中国就如生着传染病的病人一般,自己生了病,还会将病传到别人身上去,这倒是一种特别的本领。

    我们为甚么能够同化蒙古人和满洲人呢?是因为他们的文化比我们的低得多。倘使别人的文化和我们的相敌或更进步,那结果便要大不相同了。

    现在听说又有别国人在尊重中国的旧文化了,那里是真在尊重呢,不过是利用!

    旧文章,旧思想,都已经和现社会毫无关系了。

    有些读书人说,我们看这些古东西,倒并不觉得于中国怎样有害,又何必这样决绝地抛弃呢?是的。然而古老东西的可怕就正在这里。倘使我们觉得有害,我们便能警戒了,正因为并不觉得怎样有害,我们这才总是觉不出这致死的毛病来。因为这是``软刀子''。

    中国的文化,都是侍奉主子的文化,是用很多的人的痛苦换来的。
}

\subsection{好东西歌}
\item {
    \CJKfamily{chisugafont} 相骂声中失土地,相骂声中捐铜钱,失了土地捐过钱,喊声骂声也寂然。

    大家都是好东西,终于聚首一堂来吸雪茄烟。
}

\subsection{今春的两种感想}
\item {
    \CJKfamily{chisugafont} 中国的事情往往是招牌一挂就算成功了。

    看不懂也并非一定是看者知识太浅,实在是它根本上就是看不懂。文章本来有两种:一种是看得懂的,一种是看不懂的。假期你看不懂就自恨浅薄,那就是上当了。
}

\subsection{今春的两种感想}
\item {
    \CJKfamily{chisugafont} ``不识不知,顺帝之则。''
    
    中国的劳苦大众,从知识阶级看来,是和花鸟为一类的。
}

\subsection{上海所感}
\item {
    \CJKfamily{chisugafont} 我们从幼小以来,就受着对于意外的事情,变化非常事情,绝不惊奇的教育。

    外交家是多疑的,我却觉得中国人大抵都多疑。如果跑到乡下去,向农民问路径,问他的姓名,问收成,他总不大肯说老实话。
}

\subsection{自题小像}
\item {
    \CJKfamily{chisugafont} 灵台无计逃神矢,风雨如磐暗故园。

    寄意寒星荃不察,我以我血荐轩辕。
}

\subsection{赠邬其山}
\item {
    \CJKfamily{chisugafont} 廿年居上海,每日见中华:

    有病不求药,无聊才读书。

    一阔脸就变,所砍头渐多。

    忽而又下野,南无阿弥陀。
}

\subsection{戌年初夏偶作}
\item {
    \CJKfamily{chisugafont} 万家墨面没蒿莱,敢有歌吟动地哀。

    心事浩茫连广宇,于无声处听惊雷。
}
